\documentclass[]{article}
\usepackage{lmodern}
\usepackage{amssymb,amsmath}
\usepackage{graphicx} 
\usepackage{ifxetex,ifluatex}
\usepackage{fixltx2e} % provides \textsubscript
\usepackage{makecell}
\ifnum 0\ifxetex 1\fi\ifluatex 1\fi=0 % if pdftex
  \usepackage[T1]{fontenc}
  \usepackage[utf8]{inputenc}
\else % if luatex or xelatex
  \ifxetex
    \usepackage{mathspec}
  \else
    \usepackage{fontspec}
  \fi
  \defaultfontfeatures{Ligatures=TeX,Scale=MatchLowercase}
\fi
% use upquote if available, for straight quotes in verbatim environments
\IfFileExists{upquote.sty}{\usepackage{upquote}}{}
% use microtype if available
\IfFileExists{microtype.sty}{%
\usepackage[]{microtype}
\UseMicrotypeSet[protrusion]{basicmath} % disable protrusion for tt fonts
}{}
\PassOptionsToPackage{hyphens}{url} % url is loaded by hyperref
\usepackage[unicode=true]{hyperref}
\hypersetup{
            pdftitle={Specyfikacja funkcjonalna programu realizującego kompresję plików wykorzystującego algorytm Huffmana},
            pdfauthor={Adrian Chmiel, Mateusz Tyl},
            pdfborder={0 0 0},
            breaklinks=true}
\urlstyle{same}  % don't use monospace font for urls
\IfFileExists{parskip.sty}{%
\usepackage{parskip}
}{% else
\setlength{\parindent}{0pt}
\setlength{\parskip}{6pt plus 2pt minus 1pt}
}
\setlength{\emergencystretch}{3em}  % prevent overfull lines
\providecommand{\tightlist}{%
  \setlength{\itemsep}{0pt}\setlength{\parskip}{0pt}}
\setcounter{secnumdepth}{0}
% Redefines (sub)paragraphs to behave more like sections
\ifx\paragraph\undefined\else
\let\oldparagraph\paragraph
\renewcommand{\paragraph}[1]{\oldparagraph{#1}\mbox{}}
\fi
\ifx\subparagraph\undefined\else
\let\oldsubparagraph\subparagraph
\renewcommand{\subparagraph}[1]{\oldsubparagraph{#1}\mbox{}}
\fi

% set default figure placement to htbp
\makeatletter
\def\fps@figure{htbp}
\makeatother


\title{Specyfikacja funkcjonalna programu realizującego kompresję plików wykorzystującego algorytm Huffmana}
\author{Autorzy: Adrian Chmiel, Mateusz Tyl}
\date{08.04.2023}



\begin{document}
\maketitle
\begin{center}
Historia zmian dokumentu:\\
\end{center}

\begin{tabular}{|c|c|c|c|}
  \hline 
  Autor: & Data: & Opis zmiany:& Wersja dokumentu \\
  \hline
  Mateusz Tyl & 25.03.2023 & Dodanie pierwszego szkicu & 1.0 \\
  \hline
  Mateusz Tyl & 08.04.2023 & \makecell{Dodanie informacji o nowych \\funkcjonalnościach, poprawki} & 2.0\\
  \hline
  Mateusz Tyl & 08.04.2023 & \makecell{Dodanie informacji o debugowaniu} & 2.1\\
  \hline
  Adrian Chmiel & 10.04.2023 & \makecell{Drobne poprawki} & 2.2\\
  \hline
   Mateusz Tyl & 11.04.2023 & \makecell{Dodanie opisu słownika} & 2.3\\
  \hline
\end{tabular} 
\section{Cel projektu}\label{header-n231}

Przeznaczeniem progamu jest kompresja plików takich jak pliki tekstowe, graficzne, dźwiękowe z wykorzystaniem \emph{algorytmu Huffmanna}.\\
Program oferuje różnego rodzaju funkcjonalności na które składają się:
\begin{itemize}
\item
Kompresja danych w trybach 8, 12, 16 bitowym
\item
Możliwość zaszyfrowania danych wyjściowych
\item
Odpowiednio dostosowana struktura pliku wyjściowego w celu łatwego zaimplemetowania dekompresora
\item
Obsługa sum kontrolnych w celu sprawdzania poprawności pliku, opcjonalnie wymuszenie kompresji.\end{itemize}
\section{Teoria}\label{header-n281}
Algorytm Huffmana wykorzystany w tym programie to przykład jednego z najprostszych i najłatwiejszych w implementacji algorytmów wykorzystujących kompresję bezstratną.\\
Zasada działania opiera się na utworzeniu dla każdego znaku w pliku nieskompresowanym kodu binarnego o zmiennej długości. Kody tworzy się za pomocą drzewa binarnego, badając częstotliwości wystąpień każdego ze znaków. Dla częściej występujących znaków otrzymamy krótsze kody, dla rzadziej występujących dłuższe. Kompresując plik zamieniamy każdy znak na odpowiadający  mu kod i zapisujemy kolejno po sobie w pliku wyjściowym.\\
Przykładowo, dla pliku zawierającego tekst \texttt{alamakota} otrzymamy następujący zestaw kodów:\\
Character: k, Code: 1111\\
Character: o, Code: 1110\\
Character: l, Code: 1101\\
Character: m, Code: 1100\\
Character: t, Code: 101\\
Character: \textbackslash n, Code: 100\\
Character: a, Code: 0\\
Zatem po kompresji otrzymamy następujący ciąg danych: \\
01101011000111111101010100\\
Plik skompresowany tworzony przez program zawiera jeszcze na początku nagłówek, słownik oraz ewentualnie na końcu dodatkowe zera wypełniające brakujące bity. Więcej o tym w kolejnych rozdziałach.
\section{Dane wejściowe}\label{header-n233}
Program wymaga przekazania dowolnego pliku, który ma zostać skompresowany. Program obsługuje prawidłowo pliki tekstowe, dźwiękowe, graficzne.  \\Dalsze instrukcje zostaną przedstawione w następnym rozdziale.
\section{Argumenty wywołania programu}\label{header-n256}

Program działa w trybie wsadowym. Wymaganymi parametrami są kolejno: nazwa pliku do skompresowania, nazwa pliku wyjściowego.
\\Lista parametrów opcjonalnych:

\begin{itemize}
\item
  \texttt{-o0} - brak kompresji
\item
   \texttt{-o1} - kompresja 8 bitowa
\item
   \texttt{-o2} - kompresja 12 bitowa
\item
   \texttt{-o3} - kompresja 16 bitowa
\item
 \texttt{-h} - wyświetl pomoc do programu
\item
 \texttt{-c} - zaszyfruj wynik działania programu
\item
 \texttt{-x} - wymuś kompresję
\item
 \texttt{-DDEBUG} - włącz tryb debugowania
\end{itemize}
W przypadku niepodania argumentów opcjonalnych domyślnie ustawiona zostanie kompresja 8 bitowa wraz z wyłączonym szyfrowaniem i wymuszaniem kompresji.

\section{Przykład wywołania programu}\label{header-n233}

\begin{itemize}
\item
 ./program input output
\\Program wczyta plik o nazwie input i skompresuje go do pliku output z domyślnymi ustawieniami
\item
 ./program input output -c
\\Program wczyta plik o nazwie input i skompresuje go do pliku output z domyślnymi ustawieniami oraz włączonym szyfrowaniem
\item
 ./program input output -o3 -c
\\Program wczyta plik o nazwie input i skompresuje go do pliku output wykorzystując kompresję 16 bitową i włączone szyfrowanie 
\item
 ./program lub ./program -h
\\Zostanie wyświetlona pomoc do programu
\item
 ./program input output -DDEBUG -x
\\Program przejdzie w tryb debugowania, zostanie włączone wymuszanie kompresji

\end{itemize}

\section{Struktura pliku wyjściowego}\label{header-n279}
Pierwsze cztery bajty pliku wyjściowego zarezerwowane są na nagłówek. \\
Pierwsze dwa bajty nagłówka to pierwsze litery nazwisk autorów - CT.\\
Kolejny bajt to maska, z której można odczytać szczegółowe informacje o pliku. Ostatnim bajtem nagłówka jest wynik wyliczonej sumy kontrolnej.\\
Następnie pojawia się słownik, po nim znajdują się kolejno po sobie skompresowane dane w postaci kodów Huffmana o zmiennej długości.\\
W przypadku, gdy po zapisaniu wszystkich danych ilość bitów w ostatnim bajcie jest niekompletna, zostanie wykonane dopełnienie zerami.
\section{Struktura maski w nagłówku pliku}\label{header-n279}

    Szablon bitowy: 0bKKSZCEEE
\begin{itemize}
    \item K - sposób kompresji: 00 - brak, 01 - 8-bit, 10 - 12-bit, 11 - 16-bit
   \item  S - szyfrowanie: 0 - nie, 1 - tak
   \item  Z - zapisanie informacji, czy konieczne będzie usunięcie nadmiarowego znaku \textbackslash0 z końca pliku podczas dekompresji
   \item  C - dodatkowe sprawdzenie, czy ten plik jest skompresowany: 0 - nie, 1 - tak
   \item  E - ilość niezapisanych bitów kończących (tj. dopełniających ostatni bajt) zapisana binarnie
\end{itemize}
\section{Słownik}\label{header-n279}
Kod Huffmana każdego znaku wraz z odpowiadającym mu znakiem można odczytać ze słownika, który znajduje się od razu po nagłówku. Do odczytania słownika będzie potrzebny odpowiedni algorytm.\\
Algorytm prezentuje się następująco:\\
Tworzymy pomocnicze drzewo binarne. Rozpoczynamy analizę bitów składających się na słownik. Znajdujemy się w korzeniu drzewa. W zależności na co napotkamy analizując kolejne bity(analizujemy kolejno po dwa) robimy to co następuje:
\begin{itemize}
    \item 00 - przechodzimy po drzewie w dół do lewego syna jeżeli ten jest nieodwiedzony, w przeciwnym razie w prawo
   \item  01 - to samo co 00, ale po tym przejściu znajdziemy się w liściu - wtedy otrzymujemy kod znaku w całości, kolejne 8/12/16(w zależności od poziomu kompresji) bitów to znak którego kod otrzymaliśmy.
   \item  10 - wycofanie się do ojca
   \item  11 - koniec słownika
\end{itemize}
W przypadku napotkania na 01 możemy z drzewa odczytać cały kod Hufmanna dla konkretnego znaku. Kod czytamy od liścia w stronę korzenia. Z kolejnych 8/12/16(w zależności od poziomu kompresji) bitów możemy odczytać kodowany znak. \\
\textbf{Uwaga - zawsze po odczytaniu znaku(tzn. odczytaniu 8/12/16 bitów należy w drzewie powrócić do ojca}
\\\\Przykład: (dla kompresji 8 bitowej)
Nasz przykładowy słownik:\\
\begin{center}
\texttt{0101100010000001011001000101100001100101}
\end{center}
Analizujemy pierwsze dwa bity
\begin{center}
\texttt{\underline{01}01100010000001011001000101100001100101}
\end{center}
Zatem przechodzimy do lewego syna. Znajdujemy się w liściu, co oznacza że właśnie otrzymaliśmy znak. Odczytujemy z drzewa kod Huffmana dla aktualnego znaku. W tym przypadku jest to 0. Lewą gałąź w drzewie oznaczamy zerem, prawą jedynką.
Odczytujemy kolejne 8 bitów.Te bity to znak, którego kod Huffmana właśnie otrzymaliśmy.
\begin{center}
\texttt{01\underline{01100010}000001011001000101100001100101}
\end{center}
Otrzymaliśmy zatem pierwszy znak i odpowiadający mu kod Huffmana.\\
Znak w postaci binarnej: 01100010, w postaci dziesiętnej: 98\\
Odpowiadający mu kod Huffmana to: 0\\
Dalej postępujemy tak samo, pamiętając, że jeżeli napotkamy na \texttt{11} to słownik zostaje zakończony.
 \section{Szyfrowanie}\label{header-n281} 
Szyfrowanie pliku odbywa się za pomocą \emph{Szyfrowania Vigenère’a}. Domyślnie kluczem szyfrowania jest: 
\texttt{Politechnika\textunderscore Warszawska}

\section{Komunikaty błędów}\label{header-n281}

\begin{enumerate}
\def\labelenumi{\arabic{enumi}.}
\item
  Błąd podczas wczytywania pliku wejściowego: \texttt{Input file could not be opened!}
\item
Plik wejściowy jest pusty: \texttt{Input file is empty!}
\item
Błąd podczas wczytywania pliku wyjściowego:  \texttt{Output file could not be opened!}
\item
Pominięcie niezindentyfikowanych argumentów: \texttt{Unknown argument! (ignoring...)}
\item
Wystąpił błąd podczas alokowania pamięci: \texttt{Failed memory allocation!}
\item
Napotkano na błąd podczas zarządzania pamięcią: \texttt{Compression memory failure!}

\end{enumerate}

\section{Zwracane wartości}\label{header-n281}

Program po zakończeniu pracy zwraca wartość typu całkowitego, która może być użyteczna w przypadku identyfikacji różnego rodzaju niepowodzeń:

\begin{itemize}
\item
0 - Program zakończył się pomyślnie
\item
1 - Podano za mało argumentów
\item
2 - Błąd podczas wczytywania pliku wejściowego
\item
3 - Błąd podczas wczytywania pliku wyjściowego
\item
4 - Pusty plik wejściowy
\item
5 - Wymuszono dekompresję dla pliku, który nie może zostać zdekompresowany
\item
6 - Błąd przy alokowaniu/dealokowaniu pamięci
\item
7 - Podano nieprawidłowy szyfr przy dekompresji
\end{itemize}
Należy jednak pamiętać, że podanie nieprawidłowego szyfru może się również zakończyć powodzeniem, lecz uzyskany plik wynikowy nie będzie zgodny z oryginałem tj. plikiem przed kompresją.

\section{Tryb debugowania}\label{header-n281}
Tryb debugowania włączony poprzez podanie opcjonalnego parametru -DDEBUG podczas kompilacji programu, pozwala prześledzić działanie programu oraz wyświetlić niektóre komunikaty o błędach, niewidoczne w standardowym trybie. Makefile daje możliwość kompilacji w trybie debugowania poprzez komendę \texttt{make debug}. Obejmuje on następujące możliwości:

\begin{enumerate}
\def\labelenumi{\arabic{enumi}.}
\item
Wyświetlenie komunikatów informujących o nieprawidłowym alokowaniu/realokowaniu pamięci
\item
Wyświetlenie aktualnych ustawień kompresji pliku
\item
Wyświetlenie wygenerowanych kodów Huffmana dla każdego znaku
\item
Wyświetlenie obliczonej sumy kontrolnej
\item
Wyświetlenie aktualnego klucza szyfrowania
\item
Wyświetlenie, ile bitów końcowych zostało dopełnionych
\item
Wyświetlenie statystyk na temat skompresowanego pliku

\end{enumerate}

\end{document}
