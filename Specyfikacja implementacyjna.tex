\documentclass[]{article}
\usepackage{lmodern}
\usepackage{amssymb,amsmath}
\usepackage{ifxetex,ifluatex}
\usepackage{fixltx2e} % provides \textsubscript
\usepackage{makecell}
\ifnum 0\ifxetex 1\fi\ifluatex 1\fi=0 % if pdftex
  \usepackage[T1]{fontenc}
  \usepackage[utf8]{inputenc}
\else % if luatex or xelatex
  \ifxetex
    \usepackage{mathspec}
  \else
    \usepackage{fontspec}
  \fi
  \defaultfontfeatures{Ligatures=TeX,Scale=MatchLowercase}
\fi
% use upquote if available, for straight quotes in verbatim environments
\IfFileExists{upquote.sty}{\usepackage{upquote}}{}
% use microtype if available
\IfFileExists{microtype.sty}{%
\usepackage[]{microtype}
\UseMicrotypeSet[protrusion]{basicmath} % disable protrusion for tt fonts
}{}
\PassOptionsToPackage{hyphens}{url} % url is loaded by hyperref
\usepackage[unicode=true]{hyperref}
\hypersetup{
            pdftitle={Specyfikacja implementacyjna programu realizującego kompresję plików wykorzystującego algorytm Huffmana},
            pdfauthor={Adrian Chmiel, Mateusz Tyl},
            pdfborder={0 0 0},
            breaklinks=true}
\urlstyle{same}  % don't use monospace font for urls
\IfFileExists{parskip.sty}{%
\usepackage{parskip}
}{% else
\setlength{\parindent}{0pt}
\setlength{\parskip}{6pt plus 2pt minus 1pt}
}
\setlength{\emergencystretch}{3em}  % prevent overfull lines
\providecommand{\tightlist}{%
  \setlength{\itemsep}{0pt}\setlength{\parskip}{0pt}}
\setcounter{secnumdepth}{0}
% Redefines (sub)paragraphs to behave more like sections
\ifx\paragraph\undefined\else
\let\oldparagraph\paragraph
\renewcommand{\paragraph}[1]{\oldparagraph{#1}\mbox{}}
\fi
\ifx\subparagraph\undefined\else
\let\oldsubparagraph\subparagraph
\renewcommand{\subparagraph}[1]{\oldsubparagraph{#1}\mbox{}}
\fi

% set default figure placement to htbp
\makeatletter
\def\fps@figure{htbp}
\makeatother


\title{Specyfikacja funkcjonalna programu realizującego kompresję plików wykorzystującego algorytm Huffmana}
\author{Autorzy: Adrian Chmiel, Mateusz Tyl}
\date{08.04.2023}



\begin{document}
\maketitle
\begin{center}
Historia zmian dokumentu:\\
\end{center}

\begin{tabular}{|c|c|c|c|}
  \hline 
  Autor: & Data: & Opis zmiany:& Wersja dokumentu \\
  \hline
  Mateusz Tyl & 08.04.2023 & Dodanie pierwszego szkicu & 1.0 \\
\end{tabular} 
\section{Cel dokumentu}\label{header-n231}

Celem tego dokumentu jest przedstawienie informacji o sposobie działania programu od strony technicznej. Zostanie omówiony każdy plik kodu programu oraz pliki Makefile i CMake.
\section{Informacje ogólne}\label{header-n231}
Program został napisany w całości w języku C. Wykorzystano także język powłoki bash oraz skrypty Makefile i CMake do implementacji funkcjonalności pomocniczych.\\
Program jest przystosowany do pracy i kompilacji na systemach Unixowych.\\
Do dyspozycji programisty przygotowano szeroki zakres testów do wykonania w celu sprawdzenia poprawności działania programu.

\end{document}
