\documentclass[]{article}
\usepackage{lmodern}
\usepackage{amssymb,amsmath}
\usepackage{ifxetex,ifluatex}
\usepackage{fixltx2e} % provides \textsubscript
\usepackage{makecell}
\ifnum 0\ifxetex 1\fi\ifluatex 1\fi=0 % if pdftex
  \usepackage[T1]{fontenc}
  \usepackage[utf8]{inputenc}
\else % if luatex or xelatex
  \ifxetex
    \usepackage{mathspec}
  \else
    \usepackage{fontspec}
  \fi
  \defaultfontfeatures{Ligatures=TeX,Scale=MatchLowercase}
\fi
% use upquote if available, for straight quotes in verbatim environments
\IfFileExists{upquote.sty}{\usepackage{upquote}}{}
% use microtype if available
\IfFileExists{microtype.sty}{%
\usepackage[]{microtype}
\UseMicrotypeSet[protrusion]{basicmath} % disable protrusion for tt fonts
}{}
\PassOptionsToPackage{hyphens}{url} % url is loaded by hyperref
\usepackage[unicode=true]{hyperref}
\hypersetup{
            pdftitle={Specyfikacja implementacyjna programu realizującego kompresję plików tekstowych wykorzystującego algorytm Huffmana},
            pdfauthor={Adrian Chmiel, Mateusz Tyl},
            pdfborder={0 0 0},
            breaklinks=true}
\urlstyle{same}  % don't use monospace font for urls
\IfFileExists{parskip.sty}{%
\usepackage{parskip}
}{% else
\setlength{\parindent}{0pt}
\setlength{\parskip}{6pt plus 2pt minus 1pt}
}
\setlength{\emergencystretch}{3em}  % prevent overfull lines
\providecommand{\tightlist}{%
  \setlength{\itemsep}{0pt}\setlength{\parskip}{0pt}}
\setcounter{secnumdepth}{0}
% Redefines (sub)paragraphs to behave more like sections
\ifx\paragraph\undefined\else
\let\oldparagraph\paragraph
\renewcommand{\paragraph}[1]{\oldparagraph{#1}\mbox{}}
\fi
\ifx\subparagraph\undefined\else
\let\oldsubparagraph\subparagraph
\renewcommand{\subparagraph}[1]{\oldsubparagraph{#1}\mbox{}}
\fi

% set default figure placement to htbp
\makeatletter
\def\fps@figure{htbp}
\makeatother


\title{Specyfikacja implementacyjna programu realizującego kompresję plików wykorzystującego algorytm Huffmana}
\author{Autorzy: Adrian Chmiel, Mateusz Tyl}
\date{08.04.2023}



\begin{document}
\maketitle
\begin{center}
Historia zmian dokumentu:\\
\end{center}

\begin{tabular}{|c|c|c|c|}
  \hline 
  Autor: & Data: & Opis zmiany:& Wersja dokumentu \\
  \hline
  Mateusz Tyl & 08.04.2023 & Dodanie pierwszego szkicu & 1.0 \\
\hline
  Mateusz Tyl & 08.04.2023 & Dodanie informacji o plikach & 2.0 \\
\hline
  Adrian Chmiel & 10.04.2023 & Drobne poprawki & 2.1 \\
\hline
  Mateusz Tyl & 11.04.2023 & Dodatkowe informacje do plików & 2.2 \\
\hline
\end{tabular} 
\section{Cel dokumentu}\label{header-n231}

Celem tego dokumentu jest przedstawienie informacji o sposobie działania programu od strony technicznej. Zostanie omówiony każdy plik kodu programu oraz pliki Makefile i CMake.
\section{Informacje ogólne}\label{header-n231}
Program został napisany w całości w języku C. Wykorzystano także język powłoki bash oraz skrypty Makefile i CMake do implementacji funkcjonalności pomocniczych.\\
Program jest przystosowany do pracy i kompilacji na systemach Unixowych.\\
Do dyspozycji programisty przygotowano szeroki zakres testów do wykonania w celu sprawdzenia poprawności działania programu.
W programie zaimplementowano kompresor oraz dekompresor służący do przetestowania poprawności działania kompresora.

\section{Omówienie kodu źródłowego}\label{header-n231}
Program oferuje kilka modułów, które współpracują ze sobą. Niektóre moduły pracują niezależnie od innych.

\begin{itemize}
\section{Folder "gen"}\label{header-n231}
\item
gen.c -> Program pomocniczy służący do generowania gotowych testów do tego kompresora. Program generuje losowe znaki, zależnie od ustawień podanych przez użytkownika i zapisuje je w pliku.\\ Lista argumentów: [nazwa pliku wyjściowego] [ilość znaków] [ilość różnych znaków, które mogą wystąpić w outpucie] [seed]
\item
testhufftree.c -> Program pomocniczy odpowiedzialny za wygenerowanie testu w wyniku którego możliwe było przeciążenie pamięci przez tworzenie drzewa Huffmana. Jedynym argumentem jest nazwa pliku wyjściowego
\item
testmax.c -> Program pomocniczy odpowiedzialny za wygenerowanie testu o wielkości ok. 1.5 MB. Argumentem jest nazwa pliku wyjściowego
\item
testo3.c -> Program pomocniczy odpowiedzialny za wygenerowanie testu testo3.in zawierającego wszystkie możliwe znaki/słowa dla wszystkich możliwych poziomów kompresji
\section{Folder "src"}\label{header-n231}
\item
alloc.* -> zawiera funkcje służące do bezpiecznej alokacji pamięci. Zwraca błędy jeżeli alokacja się nie powiedzie.
\item
bits\textunderscore analyze.* -> zawiera funkcje służące do analizy kolejnych bitów zawartych w kolejnych znakach przy dekompresji. Sprawdza, czy alokacja pamięci się powiodła i wypisuje stosowny komunikat w przypadku błędu. Analizuje zapisany słownik w pliku. Funkcje korzystają z pomocniczej struktury dtree odwzorowującej drzewo binarne. W pliku nagłówkowym zdefiniowany jest enum, który pełni funkcję zmiennej przechowującej aktualny tryb analizy aktualnych bitów. W zależności od odczytywanych bitów zmienna przyjmuje odpowiedni tryb i przekazuje programowi, co ma robić. Tryby są następujące: dictRoad - tryb uaktywnia się, gdy przemieszczamy się w drzewie, dictWord - uaktywniany, gdy znajdziemy się w liściu w celu odczytania znaku/słowa, bitsToWords - służy do odczytywania skompresowanych danych po odczytaniu całości słownika
\item
countCharacters.* -> zawiera funkcje odpowiedzialne za zliczanie wystąpień wszystkich znaków w pliku wejściowym. W pliku nagłówkowym zdefiniowano strukturę count\_t służącą do przechowywania ilości wystąpień każdego ze znaków oraz drzewa Huffmana. Struktura zawiera w sobie zmienną zliczającą znaki, zmienną przechowującą dany znak oraz trzy wskaźniki na struktury odpowiedzialne za realizację drzewa Huffmana. Plik źródłowy zawiera funkcje do dodawania nowych znaków do listy, sortowanie, zwalnianie pamięci, wyświetlanie listy.
\item
decompress.* -> zawiera funkcje odpowiedzialne za dekompresję. Funkcja compress jest główną funkcją dekompresującą, przyjmuje dwa argumenty - nazwa pliku wejściowego, nazwa pliku wyjściowego. Funkcja compareBuffer sprawdza, czy aktualny fragment kodu w buforze odpowiada jakiejś literze. Jeżeli tak, to zapisuje tę literę do podanego pliku.
\item
dtree.* -> zawiera typ pomocniczego pseudodrzewa do dekompresji oraz związane z nim funkcje
\item
huffman.* -> zawiera główne funkcje realizujące algorytm Huffmana. Funkcje odpowiedzialne są za generację kodów Huffmana dla każdego znaku w pliku wejściowym. Tworzone drzewo Huffmana wykorzystuje strukturę count\_t. Zaimplementowano tworzenie słownika na bazie drzewa Huffmana oraz ogólnej struktury kompresowanego pliku.  Więcej szczegółów na ten temat w specyfikacji funkcjonalnej.
\item
list.* -> zawiera implementację listy liniowej służącej do przechowywania kodów znaków oraz funkcje służące do jej obsługi.
\item
main.c -> odpowiada za sprawdzanie danych wejściowych oraz sterowanie całym programem
\item
noCompress.* -> zawiera funkcje obsługujące plik w wypadku, gdy nie zachodzi żadna kompresja
\item
output.* -> zawiera funkcje realizujące zapis skompresowanego pliku w zależności od wybranego poziomu kompresji przez użytkownika.
\item
utils.* -> zawiera pomocnicze typy oraz funkcje, obsługę komunikatów wyświetlanych użytkownikowi. Zdefiniowany jest tutaj union pack pack\_t służący jako bufor dla znaków. 
\section{Inne pliki w głównym katalogu}\label{header-n231}
test -> katalog zawierający przykładowe pliki do przetestowania, w tym pliki dźwiękowe i graficzne
\item
CMake.sh -> skrypt automatyzujący kompilację przy pomocy CMake
\item
CMakeLists.txt -> zawiera komendy do kompilacji przy użyciu CMake
\item
compare.c -> służy do porównywania, czy dwa podane pliki są jednakowe (weryfikacja, czy otrzymaliśmy plik początkowy po wykonaniu kompresji, a następnie dekompresji) Użycie: ./compare <file1> <file2>
\item
encoding.sh -> skrypt sprawdzający na wiele sposobów kodowanie danego pliku (może wymagać instalacji dodatkowych pakietów). Informacje na temat instalacji wymaganych pakietów są zawarte w tym pliku.
\item
Makefile -> zawiera komendy do standardowej kompilacji oraz przykładowe testy programu
\item
README.md -> plik do wstawienia informacji na githuba

\end{itemize}

\section{CMake}\label{header-n231}
Przekazujemy do dyspozycji programisty skrypt CMake służący do automatyzacji procesu kompilacji. W katalogu głównym programu znajdują się do tego przeznaczone pliki: CMake.sh i CMakeLists.txt.
\section{Makefile}\label{header-n231}
W pliku Makefile znajdują się różnego rodzaju testy do sprawdzenia poprawności działania programu.
Testów jest 14, każdy z nich jest oznaczony w następujący sposób: testX, gdzie X to numer testu od 1 do 14.
\\Listę testów wraz z ich opisem przedstawiono poniżej:
\begin{itemize}
\item
test1 - Przypadek, gdy plik wejściowy jest pusty
\item
test2 - Próba dekompresji uszkodzonego pliku skompresowanego
\item
test3 - Test kompresji całego tekstu Pana Tadeusza
\item
test4 - Test kompresji i dekompresji z różnymi szyframi
\item
test5 - Test kompresji i dekompresji muzyki z szyfrowaniem
\item
test6 - Test kompresji i dekompresji zdjęcia z szyfrowaniem
\item
test7 - Test wygenerowany przy pomocy gen/gen.c (kompresja 8-bit)
\item
test8 - Test wygenerowany przy pomocy gen/gen.c (kompresja 12-bit)
\item
test9 - Test wygenerowany przy pomocy gen/gen.c (kompresja 16-bit)
\item
test10 - Test na rozbudowane drzewo Huffmana
\item
test11 - Wymuszenie dekompresji dla pliku nieskompresowanego
\item
test12 - Przypadek, gdy plik nie istnieje
\item
test13 - Test na brak kompresji i brak szyfrowania
\item
test14 - Test na brak kompresji z szyfrowaniem\\
\end{itemize}
Przeprowadzenie wszystkich testów za jednym razem możemy wykonać komendą \texttt{make test}

\end{document}